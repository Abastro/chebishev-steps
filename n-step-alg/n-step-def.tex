\documentclass{article}
\usepackage{xcolor}
\usepackage{tikz-cd}
\usepackage{amsmath}
\usepackage{amssymb}
\usepackage{amsthm}
\usepackage{enumitem}
\usepackage{mathtools}
\usepackage{graphicx}
\usepackage{wrapfig}
\usepackage[a4paper, total={6in, 8in}]{geometry}

% environments
\theoremstyle{definition}
\newtheorem*{definition}{Definition}
\newtheorem*{example}{Example}
\newtheorem{exercise}{Exercise}

\theoremstyle{plain}
\newtheorem{theorem}{Theorem}[section]
\newtheorem{corollary}[theorem]{Corollary}
\newtheorem{proposition}{Proposition}[section]
\newtheorem{lemma}{Lemma}[section]

\theoremstyle{remark}
\newtheorem*{remark}{Remark}
\newtheorem*{fact}{Fact}

\newenvironment{claim}[1]{\par\underline{Claim:}\space#1}{\par\smallskip}
\newenvironment{acase}[1]{\par\underline{Case\space#1:}\space}{\par\smallskip}
\newcommand{\astep}[2]{\par\textbf{Step #1: #2}\par\smallskip}

\numberwithin{equation}{section}
% commands
\newcommand{\contra}{\Rightarrow\!\Leftarrow}

\newcommand{\bZ}{\mathbb{Z}}
\newcommand{\bQ}{\mathbb{Q}}
\newcommand{\bR}{\mathbb{R}}
\newcommand{\bC}{\mathbb{C}}

\newcommand{\rangewith}[4]{{#2}#1{#3}, \cdots, {#2}#1{#4}}
\newcommand{\funtyp}[3]{#1: & #2 & \rightarrow & #3 \\ }
\newcommand{\fundecl}[2]{& #1 & \mapsto & #2 \\ }

\title{N-step def}
\author{Junho Lee}

\begin{document}
\maketitle

\def\ms{\mathbf{m}}
\def\ns{\mathbf{n}}
\def\nsL{\mathbf{n}_L}
\def\nsR{\mathbf{n}_R}

\begin{lemma}
  Let $D_k \subset \bar{\bZ}^k$ be the set of sequences $\ns = (n_k, \cdots, n_1)$
  where any consecutive proper subsequence of finite integers $\ms = (n_{i+j}, \cdots, n_{i+1})$ satisfy $[\ms] \neq \infty$.
  Then, such $D_k$ is compact, and $n \mapsto [n]$ is contiuously well-defined on $D_k$.
  Furthermore, we have $[\nsL, \infty, \nsR] = [\nsL]$.
\end{lemma}
\begin{proof}
  First, observe that $W_j = \{ \ms \in \bZ^j \mid [\ms] = \infty \}$ is open in $\bar{\bZ}^j$:
  It is open in $\bZ^j$ by discrete topology, which itself is an open set in $\bar{\bZ}^j$.
  Since
  \[ D_k = \bigcap_{j < k, \, i + j \leq k} \bZ^{k-i-j} \times (\bZ^j \setminus W_j) \times \bZ^i, \]
  $D_k$ is closed in compact set $\bZ^k$, so $D_k$ is compact.

  Now, we show the well-definedness on $D_k$ along with $[\nsL, \infty, \nsR] = [\nsL]$ by induction on $k$.
  This is clear for $k = 1$.
  For $k > 1$, let $n = (n_1, \nsR) \in D_k$.
  Note that $\nsR \in D_{k-1}$ by definition, so we may write
  \[
    [\ns] = [n_k, \nsR] = [n_k + [\nsR]].
  \]
  Suffices to show $[\nsR] \neq \infty$.
  If $\nsR$ does not contain $\infty$, this is clear by $n \in D_k$.
  Otherwise, write $\nsR = (\mathbf{n}_{RL}, \infty, \mathbf{n}_{RR})$ where $\mathbf{n}_{RL}$ does not contain $\infty$.
  Then, $[\nsR] = [\mathbf{n}_{RL}, \infty, \mathbf{n}_{RR}] = [\mathbf{n}_{RL}] \neq \infty$
  by $\nsR \in D_{k-1}$ and induction hypothesis.
  Consequently, such $[\ns]$ well-defined on $D_k$.

  Finally, we can check by computation that
  \[
    [\nsL, \infty, \nsR] = [\nsL + [\infty + [\nsR]]] = [\nsL + [\infty]] = [\nsL]
  \]
  holds.
\end{proof}

\end{document}