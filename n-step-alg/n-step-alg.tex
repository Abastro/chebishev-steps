\documentclass{article}

\usepackage{xcolor}
\usepackage{tikz-cd}
\usepackage{amsmath}
\usepackage{amssymb}
\usepackage{amsthm}
\usepackage{enumitem}
\usepackage{mathtools}
\usepackage{graphicx}
\usepackage{wrapfig}
\usepackage[a4paper, total={6in, 8in}]{geometry}

% environments
\theoremstyle{definition}
\newtheorem*{definition}{Definition}
\newtheorem*{example}{Example}
\newtheorem{exercise}{Exercise}

\theoremstyle{plain}
\newtheorem{theorem}{Theorem}[section]
\newtheorem{corollary}[theorem]{Corollary}
\newtheorem{proposition}{Proposition}[section]
\newtheorem{lemma}{Lemma}[section]

\theoremstyle{remark}
\newtheorem*{remark}{Remark}
\newtheorem*{fact}{Fact}

\newenvironment{claim}[1]{\par\underline{Claim:}\space#1}{\par\smallskip}
\newenvironment{acase}[1]{\par\underline{Case\space#1:}\space}{\par\smallskip}
\newcommand{\astep}[2]{\par\textbf{Step #1: #2}\par\smallskip}

\numberwithin{equation}{section}
% commands
\newcommand{\contra}{\Rightarrow\!\Leftarrow}

\newcommand{\integer}{\mathbb{Z}}
\newcommand{\ratio}{\mathbb{Q}}
\newcommand{\real}{\mathbb{R}}
\newcommand{\complex}{\mathbb{C}}

\newcommand{\rangewith}[4]{{#2}#1{#3}, \cdots, {#2}#1{#4}}
\newcommand{\funtyp}[3]{#1: & #2 & \rightarrow & #3 \\ }
\newcommand{\fundecl}[2]{& #1 & \mapsto & #2 \\ }

\newcommand{\abs}[1]{\left\lvert{#1}\right\rvert}

\title{N-step decision algorithms}
\author{Junho Lee}

\begin{document}
\pagecolor{white}
\color{black}
\maketitle

It is a hard problem to determine if a number is relational or not.
Algorithmic approach could be of help here -
while there is no way to directly decide relational property,
one could decide if a number is a root of chebyshev polynomial of certain steps.
This write-up suggests multiple algoirthms to perform this operation.

\section{Problem Definition}

Given $u$, we are to find the minimal $k$ where $u$ is the root of $s^{\mathbf{n}}_k$.
For this purpose, it helps to consider $s^{\mathbf{n}}_k(u)$ as a function of $\mathbf{n} \in \integer^{k-1}$.
Hence, we will fix $u$ and denote the chebyshev polynomial as $s_k(\mathbf{n})$ or $s(\mathbf{n})$.

Under this setting, we are tasked to determine if a multivariate polynomial $s_k(\mathbf{n})$
has a zero within $\integer^{k-1}$.

\subsection{Prior consideration}

Recall that when $\mathbf{n} = (\mathbf{n}_L, n_i, \mathbf{n}_R)
  = (\mathbf{n}_{L'}, n_{i-1}, n_i, n_{i+1}, \mathbf{n}_{R'})$,
we have
\begin{equation}\label{key_equation}
  s(\mathbf{n}) =
  n_i u \cdot s(\mathbf{n}_L) \cdot s(\mathbf{n}_R)
  - s(\mathbf{n}_{L'}, n_{i-1}, n_{i+1}, \mathbf{n}_{R'}).
\end{equation}

This implies the following properties:
\begin{enumerate}
  \item When $n_i = 0$, $s(\mathbf{n}) = - s(\mathbf{n}_{L'}, n_{i-1}, n_{i+1}, \mathbf{n}_{R'})$.
  Hence, if such $\mathbf{n}$ is a zero, there is a shorter sequence which is a zero.
  \item As $n_i \to \infty$, $s(\mathbf{n}) / (n_i u)$ converges to $s(\mathbf{n}_L) \cdot s(\mathbf{n}_R)$.
\end{enumerate}

The first property implies that we do not need to check $n_i = 0$ case for a minimal polynomial,
so our search is reduced to finding a root in $\integer^{k-1} \setminus \{0\}$.

On the other hand, the latter property means the monic polynomial
\[
  \tilde{s}_k(\mathbf{n}) := \frac{s_k(\mathbf{n})}{u^{k-1} n_1 \cdots n_{k-1}}
\]
converges to the product of another such monic polynomials $\tilde{s}_i$ and $\tilde{s}_{k-i}$.
This indicates that this monic polynomial has a limiting behavior
based on chebyshev polynomials of lower degree.

\section{Recursive computation of minimum of $\tilde{s}_k$}

\subsection{Limit points of range of $\tilde{s}_k$}

First, we capture the general behavior of a function with domain $\integer^{k-1} \setminus \{0\}$.
To simplify the logic, we may reparameterize $\tilde{s}_k(\mathbf{n})$
by $\mathbf{x}$ where $x_i = 1 / n_i$.
This is simply inversion of each parameter.
Under this terms, $\mathbf{x} \in T^{k-1}$ and $\tilde{s} : T^{k-1} \to \real$
where $T := \{ 1 / k \mid k \in \integer - \{ 0 \} \}$.
Clearly, such $\tilde{s}$ is a continuous function.
Furthermore, it is easy to see that $\tilde{s}$ is defined on the case where some $x_i$'s are 0.
Hence we may regard $\tilde{s} \in C(\bar{T}^{k-1})$.

Then, we have the following property which holds for any function in $C(\bar{T}^{k-1})$.
\begin{proposition}
  Suppose there is a sequence $\{ \mathbf{x}^\alpha \} \subset \bar{T}^{k-1}$
  where each $\mathbf{x}_\alpha$ is distinct
  and $\tilde{s}(\mathbf{x}^\alpha)$ converges to some $c \in \real$.
  Then, $c$ is in the limit point $\tilde{s}(\lim T^{k-1})$.
\end{proposition}
\begin{proof}
  Notice that $\bar{T} = T \cup \{ 0 \}$ is a closed subset of compact interval $[-1, 1]$,
  so $\bar{T}$ should be compact. Then, $T^{k-1}$ should be compact as its product.
  Thus, for the given sequence $\{ \mathbf{x}^\alpha \}$,
  there is a subsequence $\{ \mathbf{y}^\beta \}$ which converges to some $\mathbf{y} \in \bar{T}^{k-1}$.
  As each $\mathbf{y}_\beta$ should be distinct as well, one has $\mathbf{y} \in \lim T^{k-1}$.
  Then, by continuity of $\tilde{s}$ we get $\tilde{s}(\mathbf{y}^\beta) \to \tilde{s}(\mathbf{y})$.
  Therefore, $c = \tilde(s)(\mathbf{y}) \in \tilde{s}(\lim T^{k-1})$.
\end{proof}

\begin{corollary}
  There are finitely many distinct $\mathbf{x}$ where $\tilde{s}(\mathbf{x}) = c$ holds.
  Specifically for $c = 0$, $\tilde{s}$ has finitely many distinct zeros.
\end{corollary}

\begin{corollary}\label{image_compact}
  The image $\tilde{s}(\bar{T}^{k-1})$ is closed and compact (this follows from $T^{k-1}$ being compact).
\end{corollary}

By Corollary \ref{image_compact}, one can find a $\mathbf{x} \in \bar{T}^{k-1}$
which achieves the minimal distance between $\tilde{s}(\bar{T}^{k-1})$ and $0$.
In particular, the zeros are obtained as such $\mathbf{x}$ if and only iff this distance is $0$.
Here, we could get an idea on how to find zeros of $\tilde{s}$.

\subsection{Overview of the algorithm}

This algorithm is to find the minimal distance $\epsilon$ between $\tilde{s}(\bar{T}^{k-1})$ and $0$,
and the value $\mathbf{m} \in \bar{T}^{k-1}$ which achieves the minimum.
The algorithm finds this inductively,
assuming the minimal distance $\epsilon_j$ and the argmin $\mathbf{m}^j$ is found for all $j < k$.
We may also assume that the minimal distance $\epsilon_j$ is nonzero,
i.e. $\tilde{s}_j$ does not have a root.

The crux of the algorithm is to restrict number of candidate $\mathbf{x}$'s,
so that the minimum could be searched and found among them in a manageable time.

\subsection{Bounds of $x_i$}

Suppose $i$ is fixed.
Recall the equation \eqref{key_equation}.
By normalizing, the following can be obtained.

\[
  \tilde{s}(\mathbf{x}) = \tilde{s}(\mathbf{x}_L, x_i, \mathbf{x}_R) =
  \tilde{s}(\mathbf{x}_L) \cdot \tilde{s}(\mathbf{x}_R)
  - \frac{x_i}{u^2}
  (x_{i+1} \cdot \tilde{s}(\mathbf{x}_L) \cdot \tilde{s}(\mathbf{x}_{R'})
  + x_{i-1} \cdot \tilde{s}(\mathbf{x}_{L'}) \cdot \tilde{s}(\mathbf{x}_R))
\]

We may substitute the "constant" and "slope" term as
\begin{align*}
  & A(x_L, x_R) := \tilde{s}(\mathbf{x}_L) \cdot \tilde{s}(\mathbf{x}_R),\\
  & B(x_L, x_R) := \frac{1}{u^2}
  (x_{i+1} \cdot \tilde{s}(\mathbf{x}_L) \cdot \tilde{s}(\mathbf{x}_{R'})
  + x_{i-1} \cdot \tilde{s}(\mathbf{x}_{L'}) \cdot \tilde{s}(\mathbf{x}_R))
\end{align*}
so that we have $\tilde{s}(\mathbf{x}) = A(\mathbf{x}_L, \mathbf{x}_R) - x_i B(\mathbf{x}_L, \mathbf{x}_R)$.
Remark that this is a linear polynomial in terms of $x_i$.

From the above equation, we can deduce
\[
  \abs{x_i} = \frac{\abs{A(\mathbf{x}_L, \mathbf{x}_R) - \tilde{s}(\mathbf{x})}}{\abs{B(\mathbf{x}_L, \mathbf{x}_R)}}
  \geq \frac{\abs{\abs{A(\mathbf{x}_L, \mathbf{x}_R)} - \abs{\tilde{s}(\mathbf{x})}}}{\abs{B(\mathbf{x}_L, \mathbf{x}_R)}}
\]
and given the minimum $\abs{\tilde{s}(\mathbf{m})} = \epsilon \leq \delta
< \min_{\mathbf{x}_L, \mathbf{x}_R} \abs{A(\mathbf{x}_L, \mathbf{x}_R)}$,
\begin{equation}\label{normal_key_condition}
  \abs{m_i}
  \geq \frac{\abs{A(\mathbf{m}_L, \mathbf{m}_R)} - \abs{\tilde{s}(\mathbf{m})}}{\abs{B(\mathbf{m}_L, \mathbf{m}_R)}}
  \geq \frac{\min \abs{A(\mathbf{x}_L, \mathbf{x}_R)} - \delta}{\max \abs{B(\mathbf{x}_L, \mathbf{x}_R)}}
\end{equation}

Therefore, we obtain the lower bound $L$ of $m_i$
whenever there is a "candidate minimum" $\delta \geq \epsilon$
which is smaller than minimum of $A$.
Since $n_i = 1 / m_i \leq 1 / L$, this gives the finite set of $\{ n_i \}$ to search the argmin from.

\subsection{Minimum of $A$, and finding the candidate minimum}

We start by the obvious remark regarding minimum of $A$:
\begin{remark}
  $\min \abs{A(\mathbf{x}_L, \mathbf{x}_R)}
  = \min_{\mathbf{x}_L} \abs{\tilde{s}(\mathbf{x}_L)} \cdot \min_{\mathbf{x}_R} \abs{\tilde{s}(\mathbf{x}_R)}
  = \epsilon_i \cdot \epsilon_{k-i} > 0$.
\end{remark}
By the induction hypothesis, this value could be computed, along with the argmin.

Let the $\mathbf{p}_L, \mathbf{p}_R$ be the argument
which achieves the minimum $\abs{A(\mathbf{p}_L, \mathbf{p}_R)} = \min \abs{A(\mathbf{x}_L, \mathbf{x}_R)}$.
Then for $\mathbf{p} = (\mathbf{p}_L, p_i, \mathbf{p}_R)$,
\begin{equation}\label{candidate-min}
  \tilde{s}(\mathbf{p}) = A(\mathbf{p}_L, \mathbf{p}_R) - p_i B(\mathbf{p}_L, \mathbf{p}_R).
\end{equation}

Notice that $p_i \in T$ has a limit point at $0$ from both side.
Thus, there is small enough $p_i \in T$
such that $\lvert A(\mathbf{p}_L, \mathbf{p}_R) - p_i B(\mathbf{p}_L, \mathbf{p}_R) \rvert
< \abs{A(\mathbf{p}_L, \mathbf{p}_R)}$.
Specifically, if we take $p_i \in T$ to be closest so that

\[
  \abs{p_i - \frac{A(\mathbf{p}_L, \mathbf{p}_R)}{B(\mathbf{p}_L, \mathbf{p}_R)}}
  = d \left( \bar{T}, \frac{A(\mathbf{p}_L, \mathbf{p}_R)}{B(\mathbf{p}_L, \mathbf{p}_R)} \right),
\]
then $\abs{\tilde{s}(\mathbf{p})}$, absolute value of \eqref{candidate-min}, is minimized.
In particular,
\[
  \abs{\tilde{s}(\mathbf{p}_L, p_i, \mathbf{p}_R)}
  < \abs{\tilde{s}(\mathbf{p}_L, 0, \mathbf{p}_R)} = \abs{A(\mathbf{p}_L, \mathbf{p}_R)}.
\]

Since $\epsilon \leq \abs{\tilde{s}(\mathbf{p})} < \abs{A(\mathbf{p}_L, \mathbf{p}_R)}$,
we might take $\delta$ as minimum of all such $\abs{\tilde{s}(\mathbf{p})}$ over $i$'s.
Such $\delta$ satisfies the desired condition.

\subsection{Maximum of $B$}

To apply \eqref{normal_key_condition},
it remains to compute $\max \abs{B(\mathbf{x}_L, \mathbf{x}_R)}$.
Instead of maximum, we give the upper bound (which is likely achieved often).

Given $(\mathbf{x}_L, x_i, \mathbf{x}_R) = (\mathbf{x}_{L'}, x_{i-1}, x_i, x_{i+1}, x_{i+2})$,
by definition (and $\abs{x_j} \leq 1, \forall j$)
\begin{align*}
  \abs{B(x_L, x_R)}
  & = \frac{1}{\abs{u^2}}
  \abs{x_{i+1} \cdot \tilde{s}(\mathbf{x}_L) \cdot \tilde{s}(\mathbf{x}_{R'})
  + x_{i-1} \cdot \tilde{s}(\mathbf{x}_{L'}) \cdot \tilde{s}(\mathbf{x}_R)} \\
  & \leq \frac{\abs{x_{i+1}} \abs{\tilde{s}(\mathbf{x}_L)} \abs{\tilde{s}(\mathbf{x}_{R'})}
  + \abs{x_{i-1}} \abs{\tilde{s}(\mathbf{x}_{L'})} \abs{\tilde{s}(\mathbf{x}_R)}}{\abs{u^2}} \\
  & \leq \frac{1}{\abs{u^2}} (M_i M_{k-i-1} + M_{i-1} M_{k-i}) =: N_{k, i}
\end{align*}
where $M_j \geq \max_{\mathbf{x}} \abs{\tilde{s}_j(\mathbf{x})}$.

We may compute $M_j$ inductively as
\[
  \abs{\tilde{s}_{k+1}}
  \leq \abs{\tilde{s}_k} + \abs{x_k} \abs{x_{k-1}} \abs{u^{-2}} \abs{\tilde{s}_{k-1}}
  \leq M_k + u^{-2} M_{k-1} =: M_{k+1}.
\]
Hence, the upper bound $N_{k, i}$ of $\abs{B(x_L, x_R)}$ could be computed
and put into the $\max \abs{B(\mathbf{x}_L, \mathbf{x}_R)}$ part of \eqref{normal_key_condition}.
This concludes the logical basis of the algorithm.

\section{Recursive computation of maximum of Continued fraction}

While the above method technically works,
the given bound gets progressively worse as $k$ increases.
A better algorithm could be obtained from the relations between
inductive formula \eqref{key_equation} and continued fraction.

As demonstrated by Philip Choi, we can obtain
\[
  \frac{s(\mathbf{n})}{s(\mathbf{n}_L) \cdot s(\mathbf{n}_R)}
  = n_i u - \frac{s(\mathbf{n}_{R'})}{s(\mathbf{n}_R)} - \frac{s(\mathbf{n}_{L'})}{s(\mathbf{n}_L)}
\]
and terms like $s(\mathbf{n}_{L'}) / s(\mathbf{n}_L)$ is a continued fraction.
From this, we obtain more easily optimizable recurrence relations.

\subsection{Recurrence relations on continued fraction}

For brevity, let us denote $l(\mathbf{n})$ to be the sequence excluding the last term,
i.e. $l(\mathbf{n}, n_j) = \mathbf{n}$.

\begin{definition}
  \[ G(\mathbf{n}) = G_k(\mathbf{n}) := \frac{s_k(l(\mathbf{n}))}{u s_{k+1}(\mathbf{n})} \]
\end{definition}

Then,
\[
  \frac{s(\mathbf{n})}{s(\mathbf{n}_L) \cdot s(\mathbf{n}_R)}
  = u (n_i - G(\mathbf{n}_L) - G(\overleftarrow{\mathbf{n}_R}))
\]
which gives the following recurrence relation on $G$.

\begin{proposition}\label{fraction_recurrence}
  For $\mathbf{n} = (\mathbf{n}_L, n_i, \mathbf{n}_R)
  = (\mathbf{n}_{L'}, n_{i-1}, n_i, n_{i+1}, \mathbf{n}_{R'})$, when $i < k$,
  \[
    G_k(\mathbf{n}) = \frac{
      G(\mathbf{n}_R) (n_i - G(\mathbf{n}_L)) - G(\mathbf{n}_{R'}) G(\overleftarrow{\mathbf{n}_R})
    }{ n_i - G(\mathbf{n}_L) - G(\overleftarrow{\mathbf{n}_R}) }
  \]
\end{proposition}
\begin{proof}
  To be done
\end{proof}

\begin{proposition}
  For $i = k$,
  \[
    G_k(\mathbf{n}) = \frac{1}{u^2} \cdot \frac{1}{n_k - G_{k-1}(l(\mathbf{n}))}
  \]
\end{proposition}
\begin{proof}
  
\end{proof}

\begin{remark}
  When $\mathbf{n}$ is a zero of $s_{k+1}$, i.e. $s_{k+1}(\mathbf{n}) = 0$,
  the denominator of $G_k$ is zero. In this case, we may say $G_k = \infty$.
\end{remark}

\subsection{Maximum of $\abs{G_k}$}

Notice that similar to $\tilde{s}_k$,
$G_k$ could also be parametrized in terms of $x_i = 1 / n_i$.
As such, we may view $\abs{G_k}$ as a continuous function
from $\bar{T}^{k}$ to the compactification $\real_{+}^\infty = \real_{+} \cup \{ \infty \}$,
which gives the concrete maximum $\abs{G(m)} \in \real_{+}^\infty$.
We will demonstrate that this maximum could be computed.

\begin{proposition}
  $k \mapsto \max_{\mathbf{n}} \abs{G_k(\mathbf{n})}$ is strictly increasing until $\infty$.
\end{proposition}
\begin{proof}
  Proof by induction.
\end{proof}

\begin{corollary}
  Maximum is always attained inside $T^k$.
\end{corollary}

Denote $\mathbf{m} \in (\integer \setminus \{ 0 \})^k$
be the sequence where $G(\mathbf{m})$ obtains the aforementioned maximum.

\subsection{Bound of $n_i$}

% TODO Mention attaining maximum somewhere
For $i < k$, we can get a bound of $n_i$ as following.
Denote $\Delta = n_i - G(\mathbf{n}_L)$ and let us use shorthand
$G_L = G(\mathbf{n}_L), G_R = G(\mathbf{n}_R),
G_{R'} = G(\mathbf{n}_{R'})$, and $\overleftarrow{G_R} = G(\overleftarrow{\mathbf{n}_R})$.
Then, equation in Proposition \ref{fraction_recurrence} could be rewritten as
\[
  G = \frac{G_R \Delta - G_{R'} \overleftarrow{G_R}}{\Delta - \overleftarrow{G_R}}.
\]

Reformulated in terms of $\Delta$, we obtain
\[
  \Delta = \frac{G - G_{R'}}{G - G_R} \overleftarrow{G_R}.
\]
Given $|G| \geq M > \max \abs{G_R} \geq \max \abs{G_{R'}}$, we have the following inequality.
\[
  |\Delta| = \frac{\abs{1 - G_{R'} / G}}{\abs{1 - G_R / G}} \abs{\overleftarrow{G_R}}
  \leq \frac{1 + \abs{G_{R'}} / \abs{G}}{1 - \abs{G_R} / \abs{G}} \abs{\overleftarrow{G_R}}
  \leq \frac{1 + \max \abs{G_{R'}} / M}{1 - \max \abs{G_R} / M} \max \abs{\overleftarrow{G_R}}
\]

Since $\max \abs{G_R} = \max \abs{\overleftarrow{G_R}} = \max \abs{G_{k - i}}$
and $\max \abs{G_{R'}} = \max \abs{G_{k-i-1}}$,
this gives the bound
\[
  \abs{n_i - G_L} \leq \frac{M + \max \abs{G_{k-i-1}}}{M - \max \abs{G_{k-i}}} \max \abs{G_{k-i}}.
\]
Hence, once $n_1, \cdots, n_{i-1}$ are determined,
there are only finite possible values that $n_i$ might take.

This does not cover $i = k$ case, which has distinct bound.
% TODO

Consequently, once we know $m_1, \cdots, m_{i-1}$,
we can choose $m_i$ from finitely many possible choices and check inductively.
This gives rise to the maximum-finding algorithm for $G_k$,
which in turn gives $\infty$ when the zeros of $s_{k+1}$ could be found.

\end{document}