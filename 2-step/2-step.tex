\documentclass{article}
\usepackage{xcolor}
\usepackage{tikz-cd}
\usepackage{amsmath}
\usepackage{amssymb}
\usepackage{amsthm}
\usepackage{enumitem}
\usepackage{mathtools}
\usepackage{graphicx}
\usepackage{wrapfig}
\usepackage[a4paper, total={6in, 8in}]{geometry}

% environments
\theoremstyle{definition}
\newtheorem*{definition}{Definition}
\newtheorem*{example}{Example}
\newtheorem{exercise}{Exercise}

\theoremstyle{plain}
\newtheorem{theorem}{Theorem}[section]
\newtheorem{corollary}[theorem]{Corollary}
\newtheorem{proposition}{Proposition}[section]
\newtheorem{lemma}{Lemma}[section]

\theoremstyle{remark}
\newtheorem*{remark}{Remark}
\newtheorem*{fact}{Fact}

\newenvironment{claim}[1]{\par\underline{Claim:}\space#1}{\par\smallskip}
\newenvironment{acase}[1]{\par\underline{Case\space#1:}\space}{\par\smallskip}
\newcommand{\astep}[2]{\par\textbf{Step #1: #2}\par\smallskip}

\numberwithin{equation}{section}
% commands
\newcommand{\contra}{\Rightarrow\!\Leftarrow}

\newcommand{\integer}{\mathbb{Z}}
\newcommand{\ratio}{\mathbb{Q}}
\newcommand{\real}{\mathbb{R}}
\newcommand{\complex}{\mathbb{C}}

\newcommand{\rangewith}[4]{{#2}#1{#3}, \cdots, {#2}#1{#4}}
\newcommand{\funtyp}[3]{#1: & #2 & \rightarrow & #3 \\ }
\newcommand{\fundecl}[2]{& #1 & \mapsto & #2 \\ }

\title{Decision algorithm for roots of 2-step}
\author{Junho Lee}

\begin{document}
\pagecolor{white}
\color{black}
\maketitle

\section{General Problem statement}

Given a rational number $\lambda = -u^2$ and fixed $k$ (or number of steps),
one is to determine if $u$ is a root of chebyshev polynomial $s^\mathbf{n}_k(u)$ for some sequence $\mathbf{n}$.

Our goal here is to establish a decision algorithm for such "relational numbers within fixed steps".

\section{Directions given by Professor Hyuk Kim}

The crucial part is to reduce the number of candidates $\mathbf{n}$ to be finite,
so that it could be checked by computation.

The baseline approach is to observe that when $n_1, n_2, \cdots, n_k \to \infty$,
the leading term of increases faster than any other terms, so that only $0$ could be the solution of it.



\section{My approach for 2-step polynomials}

Will discuss $s_6$ case here.

Recall that
\[
  s^\mathbf{n}_6(u) = n_1 n_2 n_3 n_4 n_5 u^5
  - (n_1 n_2 n_3 + n_1 n_2 n_5 + n_1 n_4 n_5 + n_3 n_4 n_5) u^3
  + (n_1 + n_3 + n_5) u.
\]
and by setting $m_i = \frac{1}{n_i n_{i+1}}$, we get
\[
  \frac{s_6(u)}{n_1 n_2 n_3 n_4 n_5} = u \cdot (u^4 - (m_1 + m_2 + m_3 + m_4) u^2 + (m_1 m_3 + m_1 m_4 + m_2 m_4)).
\]

Hence, for nonzero roots $u$, $\mu := -\lambda = u^2$ is a root of
\[
  P(x) = x^2 - (m_1 + m_2 + m_3 + m_4) x + (m_1 m_3 + m_1 m_4 + m_2 m_4).
\]
The idea is from the observation that \textit{the $m_i$'s only accumulation point is around $0$.}
We will give lower bounds to 3 $m_i$'s, so that the other $m_j$ is determined.
Then, there are only finitely many corresponding $n_i$'s possible,
from which we can decide existence of such a polynomial.

\subsection{First bounds: $m_2$ and $m_3$}

We can reformulate $P(x)$ to
\[
  P(x) = (x - m_1 - m_2) (x - m_3 - m_4) - m_2 m_3 = 0
\]
so that the following can be obtained.
\[
  \lvert \mu - (m_1 + m_2) \rvert \cdot \lvert \mu - (m_3 + m_4) \rvert = \lvert m_2 m_3 \rvert
\]

Recall that $m_i = 1 / (n_i n_{i+1})$, hence being a reciprocal of an integer.
Specifically,
\[
  \mu - (m_1 + m_2) \in
  R_\mu := \left\{ \mu - \frac{1}{k_1} - \frac{1}{k_2} \; \middle\vert \; k_1, k_2 \in \integer \right\}.
\]
Thus, $\lvert m_2 m_3 \rvert \in R_\mu \cdot R_\mu$,
from which one could deduce
\[ \lvert m_2 m_3 \rvert \geq \min \lvert R_\mu \cdot R_\mu \rvert = \min(R_\mu)^2. \]

To use this inequality, we need to show that $0$ is not an accumulation point of $R_\mu$.

\begin{proposition}
  All accumulation points of $\{ \frac{1}{k_1} + \frac{1}{k_2} \mid k_1, k_2 \in \integer \}$
  are $\{ \frac{1}{N} \mid N \in \integer \}$.
\end{proposition}

% TODO

This fact was missed in the seminar:
giving lower bound $\lvert m_2 m_3 \rvert \geq L$ means both $m_2$ and $m_3$ are bounded.
In particular, $\lvert m_2 \rvert, \lvert m_3 \rvert \leq 1$,
so $\lvert m_2 \rvert \geq L$ and the same holds for $m_3$ as well.

This bounds let us to choose and fix $m_2 := a_2$ and $m_3 := a_3$, so that only $m_1$ and $m_4$ are remaining.

\subsection{Remaining bounds}

Let us get our focus back into this (updated) formula.
\[
  \lvert \mu - (m_1 + a_2) \rvert \cdot \lvert \mu - (a_3 + m_4) \rvert = \lvert a_2 a_3 \rvert
\]
Some balancing is needed here:
\[
  A_\mu (m_1) \cdot B_\mu (m_4)
  := \left\lvert 1 - \frac{m_1}{\mu - a_2} \right\rvert \cdot \left\lvert 1 - \frac{m_4}{\mu - a_3} \right\rvert
  = \left\lvert \frac{a_2 a_3}{(\mu - a_2)(\mu - a_3)} \right\rvert =: D_\mu
\]
Notice that the right-hand side $D_\mu$ is a constant, since $\mu, a_2, a_3$ were fixed.

We compare $D_\mu$ with $1$.

\begin{acase}{$D_\mu > 1$}
  WLOG $A_\mu (m_1) \geq B_\mu (m_4)$, then $A_\mu (m_1) \geq \sqrt{D_\mu} > 1$.
  In this case,
  \[
    \frac{m_1}{\mu - a_2} \leq 1 - \sqrt{D_\mu} < 0 \text{ or } \frac{m_1}{\mu - a_2} \geq 1 + \sqrt{D_\mu}
  \]
\end{acase}

\begin{acase}{$D_\mu < 1$}
  WLOG $A_\mu (m_1) \leq B_\mu (m_4)$, then $A_\mu (m_1) \leq \sqrt{D_\mu} < 1$.
  This gives the bounds
  \[
    0 < 1 - \sqrt{D_\mu} \leq \frac{m_1}{\mu - a_2} \leq 1 + \sqrt{D_\mu}.
  \]
  Specifically, we obtain a lower bound $m_1 \geq (1 - \sqrt{D_\mu}) (\mu - a_2)$.
\end{acase}

\end{document}
